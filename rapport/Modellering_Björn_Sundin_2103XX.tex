\documentclass[12pt, a4paper]{article}

\usepackage[utf8]{inputenc}
\usepackage[swedish]{babel}

\usepackage{parskip}
\usepackage{setspace}
\usepackage[babel]{microtype}

\usepackage{amsmath}

\usepackage[style=apa, citestyle=apa]{biblatex}
\addbibresource{bibliography.bib}

\usepackage{svg}
\svgpath{{bilder/}}

\usepackage{dirtytalk}

\begin{document}

\begin{titlepage}
    \begin{center}
        \Huge{Simuleringsövning Fysik 3}\bigskip \\
        \LARGE \today\vspace{1cm}\\
        \large \setstretch{1.5}
        \textbf{Elev:} Björn Sundin\\
        \textbf{Handledare:} Carlo Ruberto
    \end{center}
\end{titlepage}

\section{Miljö och verktyg}
Programmeringsspråket C++20 användes för samtliga simuleringar. Biblioteket \say{matplotplusplus} (\cite{matplotplusplus}) användes för att producera graferna och \say{mp-units} (\cite{mp-units}) användes för statiskt typade fysikaliska enheter. Användandet av mp-units i projektet gjorde att alla felberäkningar med enheter fångades automatiskt av kompilatorn.

\section{Simulering av pingisboll}
Pingisbollens radie var $r=2$ cm. Massan var $m=2.7$ g. Luftmotstånds\-koefficienten antogs vara $C_D=0.3$. Startfarten var $v_0=40$ m/s och startvinkeln var $\alpha=35^\circ$. Värdet på luftdensiteten som användes var $\rho=1.225$ kg/m$^3$. Tiden som simulerades var 3 s.

Luftmotståndskraften beräknades genom $F_D=\frac{\pi}{2}r^2C_D\rho v^2$ eftersom bollen har tvärsnittsarean $A=\pi r^2$.

Flera simuleringar med olika tidssteg samt med Eulers och Euler-Cromers metod gjordes. Tidsstegen $\Delta t$ som användes var 0.1 ms, 1 ms, 5 ms, 10 ms och 20 ms. Figur \ref{fig:pingisboll} visar graferna för alla simuleringarna. De mörkare banorna visar simuleringarna med Euler-Cromers metod och de ljusare banorna visar simuleringarna med Eulers metod.

Man kan se att Eulers metod ger ett mer exakt resultat för dämpad kaströrelse jämfört med Euler-Cromers metod. Euler-Cromers metod förstärker dämpn\-ingen och bollen tappar energi fortare. Man ser det eftersom banorna konvergerar mot samma exakta lösning med lägre värden på $\Delta t$, men simuleringarna med Euler-Cromers metod avviker mer än Eulers metod för de större värdena på $\Delta t$.

För fri kaströrelse avviker Eulers och Euler-Cromers metod lika mycket men åt olika håll.

\begin{figure}[ht]
    \centering
    \includesvg[inkscapelatex=false, width=\textwidth]{pingisboll}
    \caption{Simulering av pingisboll med och utan luftmotstånd samt med Eulers och Euler-Cromers metod.}
    \label{fig:pingisboll}
\end{figure}

\section{Simulering av golfboll}
Golfbollens radie var $r=22$ mm. Massan var $m=45$ g. Luftmotstånds\-koefficienten antogs vara 0.2, något mindre än för pingisbollen eftersom ytan är skrovligare. Startfarten var $v_0=40$ m/s och startvinkeln var $\alpha=45^\circ$. Vinkelhastigheten för simuleringen med magnuseffekten var 3 varv/s motsols, alltså $\omega=6\pi$ rad/s. Tiden som simulerades var 6 s.

Dämpning av vinkelhastigheten räknades inte med i simuleringen. Den är alltså konstant i simuleringen, men i verkligheten hade den saktat ner över tid på grund av luftmotståndet.

\begin{figure}[ht]
    \centering
    \includesvg[inkscapelatex=false, width=\textwidth]{golfboll}
    \caption{Simulering av fri golfboll, golfboll med luftmotstånd och roterande golfboll med luftmotstånd.}
    \label{fig:golfboll}
\end{figure}

\printbibliography

\end{document}