\documentclass[12pt, a4paper]{article}

\usepackage[utf8]{inputenc}
\usepackage[swedish]{babel}

\usepackage{parskip}
\usepackage{setspace}
\usepackage[babel]{microtype}

\usepackage{amsmath}

\usepackage[labelsep=period, labelfont=bf, textfont=it, skip=5pt]{caption}

\usepackage[style=apa, citestyle=apa]{biblatex}
\addbibresource{bibliography.bib}

\usepackage{svg}
\svgpath{{bilder/}}

\usepackage{dirtytalk}

\begin{document}

\begin{titlepage}
    \begin{center}
        \Huge{Simuleringsövning Fysik 3}\bigskip \\
        \LARGE \today\vspace{1cm}\\
        \large \setstretch{1.5}
        \textbf{Elev:} Björn Sundin\\
        \textbf{Handledare:} Carlo Ruberto
    \end{center}
\end{titlepage}

\section{Miljö och verktyg}
Programmeringsspråket C++20 användes för samtliga simuleringar. Biblioteket \say{matplotplusplus} (\cite{matplotplusplus}) användes för att producera graferna och \say{mp-units} (\cite{mp-units}) användes för statiskt typade fysikaliska enheter. 

Användandet av mp-units i projektet gjorde att alla eventuella felberäkningar med enheter fångades automatiskt av kompilatorn innan programmet ens körs. Den här sortens statisk enhetskontroll är viktig att utnyttja i mjukvara som utför fysikaliska beräkningar - särskilt när kostnaden av felberäkningarna blir större, som i en pacemaker eller en raket.

\section{Simulering av pingisboll}
\begin{figure}[ht]
    \centering
    \caption{Simulering av pingisboll med och utan luftmotstånd samt med Eulers och Euler-Cromers metod.}
    \includesvg[inkscapelatex=false, width=\textwidth]{pingisboll}
    \label{fig:pingisboll}
\end{figure}

Pingisbollens radie var $r=2$ cm. Massan var $m=2.7$ g. Luftmotstånds\-koefficienten antogs vara $C_D=0.3$. Startfarten var $v_0=40$ m/s och startvinkeln var $\alpha=35^\circ$. Värdet på luftdensiteten som användes var $\rho=1.225$ kg/m$^3$. Tiden som simulerades var 3 s.

Luftmotståndskraften beräknades genom: 
\begin{equation*}
    \vec{F_D}=-\hat{v}\cdot\frac{\pi}{2}r^2C_D\rho|\vec{v}|^2
\end{equation*}
Eftersom bollen har tvärsnittsarean $A=\pi r^2$. $\hat{v}$ är hastighetens enhetsvektor.
Flera simuleringar med olika tidssteg samt med Eulers och Euler-Cromers metod gjordes. Tidsstegen $\Delta t$ som användes var 0.1 ms, 1 ms, 5 ms, 10 ms och 20 ms. Figur \ref{fig:pingisboll} visar graferna för alla simuleringarna. De mörkare banorna visar simuleringarna med Euler-Cromers metod och de ljusare banorna visar simuleringarna med Eulers metod.

Man kan se att Eulers metod ger ett mer exakt resultat för dämpad kaströrelse jämfört med Euler-Cromers metod. Euler-Cromers metod förstärker dämpn\-ingen och bollen tappar energi fortare. Man ser det eftersom banorna konvergerar mot samma exakta lösning med lägre värden på $\Delta t$, men simuleringarna med Euler-Cromers metod avviker mer än Eulers metod för de större värdena på $\Delta t$.

För fri kaströrelse avviker Eulers och Euler-Cromers metod lika mycket men åt olika håll.

\section{Simulering av golfboll}
\begin{figure}[ht]
    \centering
    \caption{Simulering av fri golfboll, golfboll med luftmotstånd och roterande golfboll med luftmotstånd.}
    \includesvg[inkscapelatex=false, width=\textwidth]{golfboll}
    \label{fig:golfboll}
\end{figure}
Golfbollens radie var $r=22$ mm. Massan var $m=45$ g. Luftmotstånds\-koefficienten antogs vara 0.2, något mindre än för pingisbollen eftersom ytan är skrovligare. Startfarten var $v_0=40$ m/s och startvinkeln var $\alpha=45^\circ$. Vinkelhastigheten för simuleringen med magnuseffekten var 4 varv/s motsols, alltså $\omega=6\pi$ rad/s. Tiden som simulerades var 6 s och tidssteget 1 ms användes.

Dämpning av vinkelhastigheten räknades inte med i simuleringen. Vinkelhastigheten är alltså konstant i simuleringen, men i verkligheten hade den saktat ner något över tid på grund av luftmotståndet.

Luftmotståndet beräknades på samma sätt som för pingisbollen men med golfbollens parametrar. Magnuskraften beräknades enligt: 
\begin{equation*}
    \vec{F_M}=(-v_y, v_x)\cdot2\pi\rho\omega r^3
\end{equation*}
I det vanliga kartesiska koordinatsystemet där en positiv vinkelhastighet $\omega$ innebär motsols rotering. Hastighetsvektorn roteras 90$^\circ$ motsols eftersom trycket är högre på undersidan för en boll som rör sig åt höger med en positiv vinkelhastighet. Slow motion-video av golfslag (\cite{SlowMotionGolf}) studerades för att bekräfta vinkelhastighetens riktning.

\section{Simulering av svängning}
\begin{figure}[p]
    \centering
    \caption{Simulering av fri svängning.}
    \includesvg[inkscapelatex=false, width=\textwidth]{oscillation_ratio_0_0}
    \label{fig:oscillation_0_0}
\end{figure}
\begin{figure}[p]
    \centering
    \caption{Simulering av svagt dämpad svängning med $\zeta=0.3$.}
    \includesvg[inkscapelatex=false, width=\textwidth]{oscillation_ratio_0_3}
    \label{fig:oscillation_0_3}
\end{figure}
\begin{figure}[p]
    \centering
    \caption{Simulering av kritiskt dämpad svängning med $\zeta=1$.}
    \includesvg[inkscapelatex=false, width=\textwidth]{oscillation_ratio_1_0}
    \label{fig:oscillation_1_0}
\end{figure}
\begin{figure}[p]
    \centering
    \caption{Simulering av starkt dämpad svängning med $\zeta=2$.}
    \includesvg[inkscapelatex=false, width=\textwidth]{oscillation_ratio_2_0}
    \label{fig:oscillation_2_0}
\end{figure}

Fjädermassan var $m=0.3$ kg och fjäderkonstanten $k=15$ N/m. Startpositionen var $y_0=1$ m och startfarten $v_0=0$ m/s.

\printbibliography

\end{document}